% \iffalse meta-comment
% !TEX program  = XeLaTeX
%<*internal>
\iffalse
%</internal>
%<*internal>
\fi
\begingroup
  \def\temp{LaTeX2e}
\expandafter\endgroup\ifx\temp\fmtname\else
\csname fi\endcsname
%</internal>
%<*install>

\input ctxdocstrip %

\let\MetaPrefix\relax

\declarepostamble\emptypostamble
\endpostamble

\def\MetaPrefix{-- }


\let\MetaPrefix\DoubleperCent

\generate
  {
%</install>
%<*internal>
    \usedir{source/xelatex/magicwatermark}
    \file{magicwatermark.ins} {\from{\jobname.dtx}{install}}
%</internal>
%<*install>
    \usedir{xetex/xelatex/magicwatermark}
    \file{magicwatermark.sty} {\from{\jobname.dtx}{package}}
    \nopreamble\nopostamble
    \usedir{doc/xelatex/magicwatermark}
    \file{README.md}   {\from{\jobname.dtx}{readme}}
  }

\endbatchfile
%</install>
%<*internal>
\fi
%</internal>
%<package>\NeedsTeXFormat{LaTeX2e}[2020/10/01]
%<package>\ProvidesPackage{magicwatermark}[2021/12/22 v1.0.0 watermark for ljguo ]
%<*driver>
\documentclass{ctxdoc}
\usepackage{graphicx}
\usepackage{enumitem}
\usepackage{listings}
\lstset{
frame=tb,framerule=2pt,language={[LaTeX]TeX},basicstyle={\ttfamily},
language={[LaTeX]TeX},
keywordstyle=\color{red!70!black}\bfseries,
keywordstyle={[2]\color{blue!70!black}\bfseries},
morekeywords={PageSetup,Watermark,NewWatermark,MyPageSetup,MyWatermark,EvenPageSetup,OddPageSetup,EvenWatermark,OddWatermark}}
\usepackage{fontspec}
\IfFontExistsTF{calibri.ttf}
{
  \setmainfont{calibri.ttf}[
    BoldFont = calibrib.ttf,
    ItalicFont = calibrii.ttf,
    BoldItalicFont = calibriz.ttf
  ]
}
{}
\setCJKmainfont[AutoFakeBold,AutoFakeSlant]{simfang.ttf}
\usepackage{magicwatermark}
\renewcommand*\marg[1]{\{\meta{#1}\}}
\renewcommand*\oarg[1]{[\meta{#1}]}
\renewcommand*\parg[1]{(\meta{#1})}
\begin{document}
%<!--CODEDOC-->  \DisableImplementation
  \EnableImplementation
  \DocInput{\jobname.dtx}
  \IndexLayout
  \PrintChanges
  \PrintIndex
\end{document}
%</driver>
% \fi
%

% \GetFileId{magicwatermark.sty}
%  
% \CharacterTable
%  {Upper-case    \A\B\C\D\E\F\G\H\I\J\K\L\M\N\O\P\Q\R\S\T\U\V\W\X\Y\Z
%   Lower-case    \a\b\c\d\e\f\g\h\i\j\k\l\m\n\o\p\q\r\s\t\u\v\w\x\y\z
%   Digits        \0\1\2\3\4\5\6\7\8\9
%   Exclamation   \!     Double quote  \"     Hash (number) \#
%   Dollar        \$     Percent       \%     Ampersand     \&
%   Acute accent  \'     Left paren    \(     Right paren   \)
%   Asterisk      \*     Plus          \+     Comma         \,
%   Minus         \-     Point         \.     Solidus       \/
%   Colon         \:     Semicolon     \;     Less than     \<
%   Equals        \=     Greater than  \>     Question mark \?
%   Commercial at \@     Left bracket  \[     Backslash     \\
%   Right bracket \]     Circumflex    \^     Underscore    \_
%   Grave accent  \`     Left brace    \{     Vertical bar  \|
%   Right brace   \}     Tilde         \~}
%
%
%\title{\pkg{magicwatermark}: 魔法水印宏包}
%\author{郭李军 \\ liguo1020@gmail.com}
%\def\version{v1.0.1}
%\date{\today\quad\version}
%\pagestyle{empty}
%\PageSetup*
%{
%  \begin{tikzpicture}[remember picture,overlay]
%    \node[cyan,scale = 2,shift={(-1,-1)}] at (current page.north east){--~\thepage~--};
%  \end{tikzpicture}
%}
%\Watermark
%\maketitle
%\thispagestyle{empty}
%\tableofcontents
%\begin{documentation}
%\changes{v1.0.1}{2021/12/21}{实现了一些基本功能}
%\begin{abstract}
%\pkg{magicwatermark} 是一个基于 \pkg{everypage} 和 \pkg{tikz} 宏包,并经过 \LaTeX3 \ 进行封装的宏包,它可以很灵活的设置和清除水印,所有水印内容都放在 \verb|tikz| 的一个 \verb|node| 里面,并处于页面中心。
%\end{abstract}
%\section{介绍}
%目前 \pkg{magicwatermark} 宏包只能实现如下功能
%\begin{itemize}
%  \item 设置所有页面的水印
%  \item 设置奇数页的水印
%  \item 设置偶数页的水印
%  \item 新建水印,同页面设置多个水印
%  \item 清除水印
%\end{itemize}
%\section{用户接口}
%\subsection{提供命令}
% \begin{function}[added=2021-12-21,updated=2021-12-21]{\PageSetup}
%   \begin{syntax}
%     \tn{PageSetup} \oarg{水印参数} \marg{水印内容}
%   \end{syntax}
%   此命令用于设置所有页面的水印内容,接受两个参数,第一个参数用于设置一些属性,以 \verb|key=value| 的形式给出, 第二个参数用于设置水印内容,可以是文字、图片、等.
% \end{function}
%\begin{function}[added=2021-12-21,updated=2021-12-21]{\Watermark}
%\begin{syntax}
%  \tn{Watermark}
%\end{syntax}
% 该命令用于显示上面设置的水印
%\end{function}
%\begin{function}[added=2021-12-21,updated=2021-12-21]{\EvenPageSetup}
%\begin{syntax}
% \tn{EvenPageSetup}\oarg{水印参数}\marg{水印内容}
%\end{syntax}
% 该命令用于设置偶数页的水印内容
%\end{function}
%\begin{function}[added=2021-12-21,updated=2021-12-21]{\EvenWatermark}
%\begin{syntax}
%  \tn{EvenWatermark}
%\end{syntax}
% 该命令用于显示上面设置的偶数页水印
%\end{function}
%\begin{function}[added=2021-12-21,updated=2021-12-21]{\OddPageSetup}
%\begin{syntax}
% \tn{OddPageSetup}\oarg{水印参数}\marg{水印内容}
%\end{syntax}
% 该命令用于设置奇数页的水印内容
%\end{function}
%\begin{function}[added=2021-12-21,updated=2021-12-21]{\OddWatermark}
%\begin{syntax}
%  \tn{OddWatermark}
%\end{syntax}
% 该命令用于显示上面设置的奇数页水印
%\end{function}
%\subsection{扩展命令}
%以下带星号命令用法和上面的类似,与其说是扩展,不如说是取消了限制,以上的命令都将水印内容写在一个\verb|node|中,对于对 TikZ 不太熟练的用户来说,前者更加方
%便快捷,而后者则更加灵活
%\begin{function}[added=2022-01-07,updated=2022-01-07]{\PageSetup*}
%\begin{syntax}
% \tn{PageSetup*}\marg{水印内容}
%\end{syntax}
%\end{function}
%\begin{function}[added=2022-01-07,updated=2022-01-07]{\EvenPageSetup*}
%\begin{syntax}
% \tn{EvenPageSetup*}\marg{水印内容}
%\end{syntax}
%\end{function}
%\begin{function}[added=2022-01-07,updated=2022-01-07]{\OddPageSetup*}
%\begin{syntax}
% \tn{OddPageSetup*}\marg{水印内容}
%\end{syntax}
%\end{function}
%\begin{function}[added=2021-12-21,updated=2021-12-21]{\NewWatermark}
%\begin{syntax}
% \tn{NewWatermark}\marg{水印名称} 
%\end{syntax}
%该命令用于新建一个水印,注意水印名称不能是 \verb|main|, \verb|even|, \verb|odd|.
%\end{function}
%\begin{function}[added=2021-12-21,updated=2021-12-21]{\MyPageSetup}
%\begin{syntax}
% \tn{MyPageSetup}\oarg{水印参数}\marg{水印名称}\marg{水印内容}
%\end{syntax}
%用于设置新建的水印内容
%\end{function}
%\begin{function}[added=2021-12-21,updated=2021-12-21]{\MyWatermark}
%\begin{syntax}
%  \tn{MyWatermark}\marg{水印名称}
%\end{syntax}
% 该命令用于显示上面设置的水印
%\end{function}
%\begin{function}[added=2021-12-21,updated=2021-12-21]{\ClearWatermark}
%\begin{syntax}
%  \tn{ClearWatermark}\marg{水印列表}
%该命令用于清除设置的水印,接受一个逗号列表作为参数,例如:
%\begin{itemize}
%  \item \verb|\ClearWatermark{main}| 清除由 \verb|\Watermark| 设置的水印
%  \item \verb|\ClearWatermark{even,odd}| 清除奇数和偶数页的水印
%  \item \verb|\ClearWatermark{all}| 清除所有的水印
%  \item \verb|\ClearWatermark{name1,name2,name3,...}| 
%\end{itemize}
%\end{syntax}
%\end{function}
%\subsection{参数列表}
%以上设置命令中的\textbf{水印参数}如下
%\begin{itemize}
%  \item \verb|scale = <number>| 用于设置缩放比例
%  \item \verb|opacity = <number>| 用于设置不透明度
%  \item \verb|shift = {(x,y)}| 用于按向量 (x,y) 平移
%  \item \verb|color = <color expression>| 用于设置文本颜色
%  \item \verb|rotate = <angle>| 用于设置旋转角度
% \item \verb+align = <center|left|right>+ 用于设置多行文本对齐方式,默认为 center
%  \item \verb+showframe = <true|false>+ 用于设置是否显示文本框,默认为 false
%\end{itemize}
%\subsection{一些说明}
%所有的显示命令和清除命令都仅对该命令后面的文本生效。

%\section{Examples}
%\subsection{不带星号}
%\begin{lstlisting}
\documentclass{ctexart}
\usepackage{magicwatermark}
\PageSetup[
  rotate = 30, % 设置旋转角度
  color = red!80, % 设置水印内容颜色
  scale = 6 % 设置比例
]{
  水印内容
}
\Watermark % 全文生效
\begin{document}
...
\end{document}
\end{lstlisting}
%\subsection{奇偶页}
%\begin{lstlisting}
\documentclass{ctexart}
\usepackage{magicwatermark}
\EvenPageSetup[color=cyan,showframe,scale=4]{偶数页} % 设置奇数页
\OddPageSetup[color=purple,showframe,scale=4]{奇数页} % 设置偶数页
\EvenWatermark % 显示水印
\OddWatermark
\begin{document}
...
\end{document}
\end{lstlisting}
%\subsection{带星号}
%\begin{lstlisting}
\documentclass{ctexart}
\usepackage{magicwatermark}
\PageSetup*{
  \begin{tikzpicture}[remember picture,overlay]
    \node[cyan,scale = 2,shift={(-1,-1)}] at 
    (current page.north east){--~\thepage~--};
  \end{tikzpicture}
}
\begin{document}
...
\Watermark % 该命令之前不生效
\end{document}
\end{lstlisting}
%\subsection{新建水印}
%\begin{lstlisting}
\documentclass{ctexart}
\usepackage{magicwatermark}
\NewWatermark{one} % 新建两个水印
\NewWatermark{two}
\MyPageSetup[scale=5,color=blue]{one}{水印内容一} % 设置水印
\MyPageSetup[scale=5,color=red]{two}{水印内容二}
\begin{document}
\MyWatermark{one} % 使用第一个水印
...
\newpage
\ClearWatermark{one} % 清除第一个水印
\MyWatermark{two} % 使用第二个水印
...
\end{document}
\end{lstlisting}
%\end{documentation}
%
% \StopEventually{}
% \section{代码实现}
%    \begin{macrocode}
%<*package>
%    \end{macrocode}
%  加载必要的宏包
%    \begin{macrocode}
\RequirePackage{everypage-1x,tikz,xparse,expl3}
%    \end{macrocode}
%  打开\LaTeX3 编程环境
%    \begin{macrocode}
\ExplSyntaxOn
%    \end{macrocode}  
%  声明一些变量
%    \begin{macrocode}
\tl_new:N \l_mainpage_tl
\tl_new:N \l_evenpage_tl
\tl_new:N \l_oddpage_tl
\tl_new:N \g_case_tl 
\tl_new:N \g_clear_all_tl
%    \end{macrocode}
%  定义警告
%    \begin{macrocode}
\msg_new:nnn{clear}{not~have}{I~can~not~find~the~"\clear"~watermark~!}
%    \end{macrocode}
% \begin{macro}{\PageSetup}
% 对所有页面设置水印
%    \begin{macrocode}
\NewDocumentCommand{\PageSetup}{sO{}+m}
{
    \IfBooleanTF{#1}
    {
        \tl_set:Nn \l_mainpage_tl{#3}
    }
    {
        \tl_set:Nn \l_mainpage_tl
        {
            \group_begin:
            \keys_set:nn{watermark}{#2}
            \begin{tikzpicture}[remember~picture,overlay]
                \node
                [
                    scale = \fp_use:N\l_scale_fp,
                    opacity = \fp_use:N\l_opacity_fp,
                    shift = \tl_use:N\l_shift_tl,
                    color = \tl_use:N \l_color_tl,
                    rotate = \fp_use:N \l_rotate_fp,
                    align = \tl_use:N \l_align_tl,
                    \bool_if:NTF \l_showframe_bool{draw}{},
                ]
                at(current~page.center){#3};
            \end{tikzpicture}
            \group_end:
        }
    }
}
%    \end{macrocode}
% \end{macro}
% \begin{macro}{\Watermark}
% 用于显示由 \verb|\PageSetup| 命令设置的水印
%    \begin{macrocode}
\NewDocumentCommand{\Watermark}{}
{   
    \AddEverypageHook{\tl_use:N \l_mainpage_tl}
}
%    \end{macrocode}
% \end{macro}
% \begin{macro}{\EvenPageSetup}
% 对偶数页设置水印
%    \begin{macrocode}
\NewDocumentCommand{\EvenPageSetup}{sO{}+m}
{
    \IfBooleanTF{#1}
    {
        \tl_set:Nn \l_evenpage_tl{#3}  
    }
    {
        \tl_set:Nn \l_evenpage_tl
        {
            \group_begin:
            \keys_set:nn{watermark}{#2}
            \begin{tikzpicture}[remember~picture,overlay]
                \node
                [
                    scale = \fp_use:N\l_scale_fp,
                    opacity = \fp_use:N\l_opacity_fp,
                    shift = \tl_use:N\l_shift_tl,
                    color = \tl_use:N \l_color_tl,
                    rotate = \fp_use:N \l_rotate_fp,
                    align = \tl_use:N \l_align_tl,
                    \bool_if:NTF \l_showframe_bool{draw}{},
                ]
                at(current~page.center){#3};
            \end{tikzpicture}
            \group_end:
        }
    }
}
%    \end{macrocode}
% \end{macro}
% \begin{macro}{\EvenWatermark}
% 用于显示由 \verb|\EvenPageSetup| 命令设置的水印
%    \begin{macrocode}
\NewDocumentCommand{\EvenWatermark}{}
{   
    \AddEverypageHook
    {
        \int_if_even:nT{\int_use:N \value{page}}
        {       
            \tl_use:N \l_evenpage_tl
        } 
    }
}
%    \end{macrocode}
% \end{macro}
% \begin{macro}{\OddPageSetup}
% 对奇数页设置水印 
%    \begin{macrocode}
\NewDocumentCommand{\OddPageSetup}{sO{}+m}
{
    \IfBooleanTF{#1}
    {
        \tl_set:Nn \l_oddpage_tl{#3} 
    }
    {
        \tl_set:Nn \l_oddpage_tl
    {
        \group_begin:
        \keys_set:nn{watermark}{#2}
        \begin{tikzpicture}[remember~picture,overlay]
            \node
            [
                scale = \fp_use:N\l_scale_fp,
                opacity = \fp_use:N\l_opacity_fp,
                shift = \tl_use:N\l_shift_tl,
                color = \tl_use:N \l_color_tl,
                rotate = \fp_use:N \l_rotate_fp,
                align = \tl_use:N \l_align_tl,
                \bool_if:NTF \l_showframe_bool{draw}{},
            ]
            at(current~page.center){#3};
        \end{tikzpicture}
        \group_end:
    }
    }
}
%    \end{macrocode}
% \end{macro}
% \begin{macro}{\OddWatermark}
% 用于显示由 \verb|\OddPageSetup| 命令设置的水印
%    \begin{macrocode}
\NewDocumentCommand{\OddWatermark}{}
{   
    \AddEverypageHook
    {
        \int_if_odd:nT{\int_use:N \value{page}}
        {       
            \tl_use:N \l_oddpage_tl
        } 
    }
}
%    \end{macrocode}
% \end{macro}

% \begin{macro}{\NewWatermark}
% 用于新建一个水印
%    \begin{macrocode}
\NewDocumentCommand{\NewWatermark}{m}
{
    \tl_new:c {l_#1_tl} 
    \tl_put_right:Nn \g_case_tl {{#1}{\tl_clear:c {l_#1_tl}}}
    \tl_put_right:Nn \g_clear_all_tl {\tl_clear:c {l_#1_tl}}
}
%    \end{macrocode}
% \end{macro}
% \begin{macro}{\MyPageSetup}
% 用于设置新建的水印 
%    \begin{macrocode}
\NewDocumentCommand{\MyPageSetup}{O{} m +m}
{
    \tl_set:cn {l_#2_tl}
    { 
        \group_begin:
        \keys_set:nn{watermark}{#1}
        \begin{tikzpicture}[remember~picture,overlay]
            \node
            [
                scale = \fp_use:N\l_scale_fp,
                opacity = \fp_use:N\l_opacity_fp,
                shift = \tl_use:N\l_shift_tl,
                color = \tl_use:N \l_color_tl,
                rotate = \fp_use:N \l_rotate_fp,
                align = \tl_use:N \l_align_tl,
                \bool_if:NTF \l_showframe_bool{draw}{},   
            ]
            at(current~page.center){#3};
        \end{tikzpicture}
        \group_end:
    }    
}
%    \end{macrocode}
% \end{macro}



% \begin{macro}{\MyWatermark}
% 用于显示由 \verb|\MyPageSetup| 命令设置的水印
%    \begin{macrocode}
\NewDocumentCommand{\MyWatermark}{m}
{
    \AddEverypageHook{\tl_use:c {l_#1_tl}}
}
%    \end{macrocode}
% \end{macro}
% 定义一些可选参数, 用于对水印进行设置
%    \begin{macrocode}
\keys_define:nn{watermark}
{
    scale.fp_set:N = \l_scale_fp,
    scale.initial:n = 1.0,
    scale.default:n = 1.0,
    opacity.fp_set:N = \l_opacity_fp,
    opacity.initial:n = 0.8,
    opacity.default:n = 0.8,
    shift.tl_set:N = \l_shift_tl,
    shift.initial:n = {(0,0)},
    shift.default:n = {(0,0)},
    color.tl_set:N = \l_color_tl,
    color.initial:n = black,
    color.default:n =black,
    rotate.fp_set:N =\l_rotate_fp,
    rotate.initial:n = 0.0,
    rotate.default:n = 0.0,
    align.tl_set:N = \l_align_tl,
    align.initial:n = center,
    align.default:n = center,
    showframe.bool_set:N = \l_showframe_bool,
    showframe.initial:n = false,
    showframe.default:n = true,
}
%    \end{macrocode}
% \begin{macro}{\g_clear_all_tl}
% 用于保存待清空的所有水印
%    \begin{macrocode}
\tl_gset:Nn \g_clear_all_tl
{
    \tl_clear:N \l_mainpage_tl
    \tl_clear:N \l_evenpage_tl
    \tl_clear:N \l_oddpage_tl
}
%    \end{macrocode}
% \end{macro}
% \begin{macro}{\g_case_tl}
% 用于保存一些开关
%    \begin{macrocode}
\tl_gset:Nn \g_case_tl{
        {all}
        {
          \tl_use:N\g_clear_all_tl 
        }
        {even}
        {
            \tl_clear:N \l_evenpage_tl
        }
        {odd}
        {
            \tl_clear:N \l_oddpage_tl
        } 
        {main}
        {
            \tl_clear:N \l_mainpage_tl
        }
}
%    \end{macrocode}
% \end{macro}
% \begin{macro}{\ClearWatermark}
% 用于清除设置的水印
%    \begin{macrocode}
\NewDocumentCommand{\ClearWatermark}{m}
{
\clist_gset:Nn \g_clear_clist{#1}
\clist_map_variable:NNn \g_clear_clist \clear
{  
    \exp_args:NV \str_case:nVF \clear
    \g_case_tl
    {\msg_warning:nn{clear}{not~have}}
}  
}
%    \end{macrocode}
% \end{macro}
%  关闭\LaTeX3 编程环境
%    \begin{macrocode}
\ExplSyntaxOff
%    \end{macrocode}
%    \begin{macrocode}
%</package>
%    \end{macrocode}
% \Finale
%
\endinput